\chapter{Conclusão}\label{cap:conclusao}

%Este trabalho objetivou desenvolver uma ferramenta de representação
%comportamental baseada em otimização para futebol de robôs, com meta
%intermediária de criar um modelo discreto sequencial para o problema do futebol
%de robôs.
%
%Foi desenvolvido um modelo abstrato do futebol de robôs no
%Capítulo~\ref{cap:modelagem}.  Esse modelo foi base para o programa apresentado
%no Capítulo~\ref{cap:arquitetura}.  Essa ferramenta utiliza uma combinação de
%gradiente da função objetivo, buscas aleatórias e posições chaves para
%selecionar jogadas ótimas dentro do tempo disponível para o planejamento.
%
%A ferramenta criada atingiu os objetivos desejados, gerando uma variedade de
%comportamentos através da modificação dos parâmetros da função objetivo,
%conforme mostrado no Capítulo~\ref{cap:resultados}.  Também foi desenvolvida com
%sucesso uma interface gráfica que permite modificar esses parâmetros em tempo de
%execução.
%
%O trabalho atual permite que seja criado um time superior apenas modificando os
%parâmetros da função objetivo ou adicionando novos custos à função objetivo.
%Também é possível utilizar esse modelo para aplicar outros métodos de otimização.
%
%Como trabalhos futuros, sugere-se desenvolver um sistema automático para
%realizar as partidas e avaliar o desempenho do time nessa partida.  Isso
%permitiria que algorítimos de otimização também fossem utilizados para
%melhorar os parâmetros.  Para isso, é a necessário um juiz automático
%para permitir jogos não supervisionados.

%@PÍCOLI:

O objetivo deste projeto foi desenvolver a eletrônica embarcada e o \textit{firmware} para um time de robôs autônomos, da categoria \textit{Small Size Robot League}, a fim de viabilizar a participação de uma equipe de professores e alunos em uma competição internacional de robótica, o que foi atingido com a participação na RoboCup 2017, na qual a equipe alcançou o 16º lugar, jogando 6 partidas, das quais venceu 2.

Versões preliminares do \textit{firmware} e da nova eletrônica embarcada foram desenvolvidas e testadas em campo durante a LARC 2016, aprimoradas, testadas novamente na RoboCup 2017, tendo funcionado como esperado. Entretanto, os testes mostraram que aprimoramentos devem ser feitos, por exemplo, implementar uma malha de controle de corrente a fim de evitar sobrecorrente.

Além disso, estuda-se a possibilidade de usar RTOS para executar diversas tarefas quase simultaneamente e executar o gerenciamento de recursos do hardware. 

% vim: tw=80 et ts=2 sw=2 sts=2 ft=tex spelllang=pt_br,en
